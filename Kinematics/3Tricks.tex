\subsection{General Steps}
The most difficult part about kinematics is collecting the information, and interpreting the question correctly. Here are the general steps to solve any kinematic problem:
\begin{enumerate}
  \item List down all the variables you are given, and their respective values. 
  \item Determine and list the unknown variable you want to solve for.
  \item Track down the appropriate equation that involves only the variables you listed above.
  \item Substitute in the values, and solve for the unknown.
  \item Double check your work. Do an unit analysis and check if it makes sense.
\end{enumerate}

\subsection{Free-Fall Problems}
One of the most common types of kinematic problems involve objects falling only under the force of gravity (air resistance ignored), such that their acceleration is g. The following tricks are useful in solving this type of problem:
\begin{itemize}
    \item At the peak of any trajectory, the object's y-component of velocity is zero. This is often used to calculate the peak height (by setting the velocity equal to zero).
    \item Because of conservation of energy, and as reflected mathematically in the symmetry of parabolas, an object's \emph{speed} as it passes a certain height is exactly the same on the way up as it is on the way down.
    \item Also because of the symmetry of parabolas, the magnitude of the time difference between when the object is at the peak of its trajectory ($t_{peak}$) and when it is at a particularly y-position below its peak is the same whether the object is ascending or descending at the latter position.
    \item You should memorize $g = 9.8\frac{m}{s^2}$ however for the purposes of this guide, you can use $g = 10\frac{m}{s^2}$. You can set g to be negative or positive, as long as you keep your positive direction constant throughout each question!
\end{itemize}