\begin{equation}
    F=\frac{|Q_1Q_2|}{4\pi\epsilon_0R^2}
\end{equation}
\begin{equation}
    \frac{F}{Q}=\frac{F_{any\;charge}}{Q_{any\;charge}}=E
\end{equation}
\begin{equation}
    E_{net}=E_1+E_2+E_3+\dots+E_N
\end{equation}
\begin{equation}
    V=\frac{U}{Q} \iff
    \Delta V=\frac{\Delta U}{Q}
\end{equation}

Many textbooks will make the substitution for coulomb's constant: $k=\frac{1}{4\pi\epsilon_0}$ to simplify Coulomb's Law (2.1) However, it is generally preferred to keep it in the long form as the $\pi$ may cancel out with other terms.

\subsection*{Calculus}

The only calculus equation needed is a direct restatement of (2.4). Although it has an extremely trivial derivation for any calculus student, it is vital in solving many problems involving calculus.

\begin{equation}
    E_x=-\frac{dV}{dx}
\end{equation}